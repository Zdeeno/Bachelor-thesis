\section{Introduction}

Reinforcement learning (RL) is a field of study consisting mainly of dynamic programming and machine learning. It is based on concepts of behavioural psychology, especially the trial and error method, and has in recent years experienced a rapid development due to the growth of computational power and neural networks improvement. Richard Sutton has made a helpful summary of RL concepts in his book [citace]. One of the biggest achievements was playing Atari games by an RL agent without any prior knowledge of the environment [citace]. Soon after an RL agent, able to solve simple continuous problems such as balancing inverse pendulum on a cart, was introduced. Today state-of-the-art methods can solve complex environments with infinite action spaces. The objective of this thesis is to apply these methods to control solid-state lidar sensor with a limited number of rays. The agent is divided into two parts - mapping and planning. The mapping part should create a best possible reconstruction from sparse measurements, while the planning part is focused on picking rays that will maximise reconstruction accuracy. This thesis is based on the work of Karel Zimmermann and his team [citace], which proposed a supervised learning agent for mapping and a prioritised greedy policy for planning rays [citace].
